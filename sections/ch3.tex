\chapter{Materiais e Métodos}
\label{chapther:metodologia}

\textit{\textcolor{red}{Neste capítulo, deve-se descrever qual o método empregado para atingir os objetivos e quais materiais foram necessários.}}

\section{Tópico X}
\label{section:metodologia_xxx}

\lipsum[1-1]

\subsection{Tópico X Y}
\label{subsection:metodologia_xxx_yyy}

\lipsum[1-1]

O fator de dois ($fac_2$) na Eq. \ref{eq:fac2} descreve quanto dos valores presentes na estimação realizada pela FCN pode ser considerado como um valor fora do padrão esperado de acordo com o modelo verdadeiro. 

\begin{equation}
    fac_2 =
    \begin{cases}
        1,& 0.5 \leq \frac{\hat{y}_i}{y_i} \leq 2\\
        0,& \text{sen\~ao}
    \end{cases}
    \label{eq:fac2}
\end{equation}

O objetivo é conseguir aproximar as métricas \textit{MSE} e \textit{MAE} o máximo possível de 0 ao passo que as métricas $R^2$, $r$ e $fac_2$ devem estar o mais próximo possível de 1. Em todas as equações descritas anteriormente, os parâmetros $y$ indicam a saída verdadeira, $\overline{y}$ o valor médio da saída verdadeira, $\hat{y}$ a saída estimada e $\overline{\hat{y}}$ o valor médio da saída estimada.