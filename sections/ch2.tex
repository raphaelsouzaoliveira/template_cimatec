\chapter{Revisão de Literatura}
\label{ch:revisao_literatura}

\textit{\textcolor{red}{O aluno deve abordar a literatura sobre o tema da pesquisa. Lembre-se de utilizar a literatura atual para apresentar o estado da técnica. Importante trazer uma literatura baseada em artigos. Utilizar a norma ABNT NBR 6023 para citação direta e indireta no texto.}}

\textbf{\textit{\textcolor{red}{Exemplo:}}}

Atualmente, os novos métodos de extração estão sendo investigados para substituir os métodos clássicos, como por exemplo, a extração com solvente, maceração, destilação a vácuo entre outros. Um dos mais promissores métodos é a extração com fluidos supercríticos (Supercritical Fluid Extraction – SFE), especificamente com o uso de dióxido de carbono (CO2) como o fluido supercrítico (MACHADO et al., 2013).

\textit{\textcolor{red}{Use Tabelas para fazer uma compilação de dados importantes de uma determinada área de estudo. Use ferramentas acadêmicas para ajudar na citação e formatação das referências. Exemplo: Mendelay.}}

\url{https://www.mendeley.com/newsfeed}

\textit{\textcolor{red}{Use base de dados científicas e tecnológicas para a sua pesquisa. Sugestão de bases (não se limite a essas):}}
\begin{itemize}
    \item \url{https://www.sciencedirect.com/}
    \item \url{https://www.mdpi.com/}
    \item \url{https://www.ncbi.nlm.nih.gov/pubmed/}
    \item \url{https://www.taylorfrancis.com/}
    \item \url{https://www.scielo.org/}
    \item \url{https://www.plos.org/}
    \item \url{https://worldwide.espacenet.com/}  (patentes)
\end{itemize}