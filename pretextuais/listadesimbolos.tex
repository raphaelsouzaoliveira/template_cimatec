% ABNT NBR 10719:2015
% Lista de símbolos
% Elemento opcional. Elaborada de acordo com a ordem apresentada no texto, com o devido significado.
% EXEMPLO
% dab Distância euclidiana
% O(n) Ordem de um algoritmo

%\makeatletter
%	\newcommand\criarsimbolo[2]{%
%		\write\@auxout{\noexpand\@writefile{sbl}{\noexpand\item[#1] #2}}}
%	\newcommand\imprimirlistadesimbolos{%
%		\begin{simbolos}
%			   \@starttoc{sbl}
%		\end{simbolos}}
%\makeatother
%
%\imprimirlistadesimbolos
% *****  Definição da Lista de Símbolos  *****

% \simb[entrada na lista de símbolos]{símbolo}: 
%   Escreve o símbolo no texto e uma entrada na Lista de Símbolos.
%   Se o parâmetro opcional é omitido, usa-se o parâmetro obrigatório.

\newcommand{\simb}[2][]{%
   {\addcontentsline{los}{simbolo}{\ensuremath{#2}\hspace{35pt} {#1}}}
   } 
\makeatletter
\newcommand{\listadesimbolos}{
  \pretextualchapter{Lista de Símbolos}
  {\setlength{\parindent}{0cm}
   \@starttoc{los}}}

% como a entrada será impressa
% \newcommand\l@simbolo[2]{\par #1, p.\thinspace#2} 
%\newcommand\l@simbolo[2]{\par #1 } 
%\newcommand\l@simbolo[2]{\normalsize #1 \hspace*{2.5pt}\mydots #2 \par}
%\newcommand\l@simbolo[2]{\par #1, \hfill p.\thinspace #2} 
\newcommand\l@simbolo[2]{\par #1 } 

\makeatother
% ***** fim da Lista de Símbolos *****

\listadesimbolos
% como usar
%\simb[Letra Grega]{\theta}
%\simb[coordenadas cartesianas da trajetória do corpo]{x}
%\simb[massa em Kg]{m}

