% ABNT NBR 10719:2015
% Lista de abreviaturas e siglas
% Elemento opcional. Consiste na relação alfabética das abreviaturas e siglas utilizadas no relatório, seguidas das palavras ou expressões correspondentes grafadas por extenso. Recomenda-se a elaboração de lista própria para cada tipo.
% EXEMPLO
% ABNT Associação Brasileira de Normas Técnicas
%------------------------------------------------
\cleardoublepage
\markboth{\nomname}{\nomname}

\makeatletter 
\renewcommand{\nomname}{Lista de Siglas}\@starttoc{las}
%\renewcommand*{\pagedeclaration}[1]{\dotfill \hyperpage{#1}}
\renewcommand*{\pagedeclaration}[1]{}
\newcommand{\abreviatura}[2]{\addcontentsline{las}{sig}{\numberline{#1}{#2}}}

\makeatother  

\begin{siglas}
\item[Adam] \textit{Adaptive Moment Estimation}
\item[AE] \textit{Autoencoder}
\item[AG] Algoritmo Genético
\item[CIG] \textit{Common-image Gather}
\item[CNN] \textit{Convolutional Neural Network}
\item[CRF] \textit{Conditional Random Field}
\item[ELU] \textit{Exponential Linear Unit}
\item[FCN] \textit{Fully Convolutional Network}
\item[FWI] \textit{Full Waveform Inversion}
\item[GPU] \textit{Graphical Processing Unit}
\item[Leaky ReLU] \textit{Leaky Rectified Linear Unit}
\item[MAE] \textit{Mean Absolute Error}
\item[MSE] \textit{Mean Squared Error}
\item[PCA] \textit{Principal Component Analysis}
\item[PReLU] \textit{Parametric Rectified Linear Unit}
\item[PSO] \textit{Particle Swarm Optimization}
\item[RAM] \textit{Random Access Memory}
\item[ReLU] \textit{Rectified Linear Unit}
\item[RTM] \textit{Reverse-Time Migration}
\item[SGD] \textit{Stochastic Gradient Descent}
\item[VAE] \textit{Variational Autoencoder}

\end{siglas}

\cleardoublepage
\makeatletter
\@mkboth{\MakeUppercase\nomname}{\MakeUppercase\nomname}%
\makeatother
  
\printnomenclature[3cm]